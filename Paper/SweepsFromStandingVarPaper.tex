%\documentclass[authoryear,round]{tufte-book}
\documentclass[a4paper,10pt]{article}
\usepackage[T1]{fontenc}
\usepackage[utf8]{inputenc}
\usepackage{amsmath}
\usepackage{amssymb}
\usepackage{graphicx}
\usepackage{fullpage}
\usepackage{color}
\usepackage{natbib}
\usepackage{mathrsfs}
\usepackage{array}
\newcommand{\head}[2]{\multicolumn{1}{>{\centering\arraybackslash}p{#1}}{#2}}
%\usepackage{sidecap}
\usepackage[capbesideposition={top,right},facing=yes,capbesidesep=quad]{floatrow}
\usepackage[hypertexnames=false]{hyperref}
\hypersetup{colorlinks=true, urlcolor=blue, citecolor=black, linkcolor=black}
%\usepackage{lineno}
%\usepackage{lscape}
%\usepackage{multirow}

\newcommand{\var}{\mathop{\mbox{Var}}}
\newcommand{\cov}{\mathop{\mbox{Cov}}}
\newcommand{\gc}[1]{{\it \color{red} (#1)} }
\newcommand{\jb}[1]{{\it\color{blue} (#1)} }
\def\citeapos#1{\citeauthor{#1}'s (\citeyear{#1})}
%\newcommand{\Rho}{\mathrm{P}}

%opening
\title{Theory of Sweeps from Standing Variation}
\author{
Jeremy J. Berg$^{1,2,3}$ and Graham Coop$^{1,2,3}$ \\
$^1$ Graduate Group in Population Biology, University of California, Davis. \\
$^2$ Center for Population Biology, University of California, Davis.\\
$^3$ Department of Evolution and Ecology, University of California, Davis\\
\small To whom correspondence should be addressed: \texttt{jjberg@ucdavis.edu, gmcoop@ucdavis.edu}\\
}

\date{}

\begin{document}

%\linenumbers
\maketitle

\begin{abstract}
\end{abstract}

%%%%%%%%%%%%%%%%%%%%%%%%%%%
\section{Introduction}



%%%%%%%%%%%%%%%%%%%%%%%%%%%
\section{Results}
\jb{thinking perhaps we should motivate our whole framework with a fair bit of verbal description before we delve into any math. Ideas here are reletively straightforward, but a fair bit of ``scene setting'' to do first}

The key insight for this paper is that the recombination events that occur during the period when an allele is neutral or balanced prior to a sweep can be treated like mutations, and thus under the assumption of constant population size, we can use the Ewens Sampling Formula to calculate expectations for a variety of quantities of interest in the region surrounding the selected site.

Consider a neutral locus partially linked (with recombination rate $r$) to a second locus where an allele arises, and is initially either neutral, or balanced at a low frequency, but eventually becomes beneficial (due to a change in environment). Our aim is to describe some features of the genealogy at the selected site at at adjacent neutral sites, how these features effect a number of population genetic summary statistics, and to explore the extent to which they can be used to say anything meaningful about sites where sweeps from standing variation may have occurred. 

\subsubsection{Neutral phase} \gc{not clear that neutral is a good term}
Imagine that we sample $k$ neutral alleles (lineages) from the population in the present day. 
We will assume that the selected phase of the trajectory occurs rapidly so there is no timefor coalescence to occur in the selected phase. So at the start of the neutral phase we have k lineages and the frequency of the selected allele is $f$ on the population.

We will argue that the majority of information about patterns of diversity surrounding a sweep from standing variation can be understood by understanding the genealogy at the selected site,
and where recombinations fall on this genealogy as we move away from the selected sites. 

If our allele was held fixed at low frequency $f$ over a long time period our genealogy at the selected site would simply be a coalescent where any pair of lineages at rate $ {k \choose 2}/(2 N f)$ (assuming that $f \gg 1/(2N)$ so that the standard coalescent assumptions hold). Such a fixed frequency could result from a beneficial allele that was balanced at low frequency, by strong constant selection. If instead of being held fixed at a frequency $f$ our allele was allowed to drift neutrally this constant pairwise coalescent rate would no long be appropriate as the frequency could have changed to $f^{\prime} \neq f$, and so now our instantaneous rate of coalescence would be  $1/(2Nf^{\prime})$. 

A number of researchers have studied the behavior of the coalescent within a neutral mutant subclass \citep{XXXX}, either conditional on the frequency of the allele in a sample or in the population.. \cite{XXX} has shown that the expected coalescent time is  $2 N f/ {k \choose 2}$ in the absence of other information, e.g. as to whether the allele is ancestral or derived or other information about the frequency of the allele in the past. However, the distribution of coalescence times is no longer exponential, and the successive coalscent times are not independent \cite{} as the time it takes a pair of lineages to coalescence contains information about the current frequency of the allele and thus about susequent coalescent events. Furthermore, when the allele is known to be derived the coalescent times have a more complicated expectation, as the allele is in expectation decreasing in frequency backward in time due to the conditioning on loss \citep{}.

Despite these complications we have found assuming that lineages to coalesce at a rate $ {k \choose 2}/(2 N f)$ and that coalescent time intervals are independent, i.e. that the allele frequency does not drift from $f$, is not a bad approximation when $f \ll 1$ regardless of whether the allele is ancestral or derived. In Supp. Figures XXX-XXX we show some comparisons of the coalescent process inbedded in a drifting allele frequency and this approximation. 

The main reason for using this approximation is that is gives us a simple well understood characture of the true process that describes the genealogy at the selected site reasonably accurately. We can use this to understand patterns of recombination as we move away from the selected site. We will again rely on the assumption that $f \ll 1$, and assume that any lineage that recombines out of the background of our beneficial allele will not recombine back into the background. As further consequence of $f \ll 1$ the coalescent times in our beneficial allele class (on a time-scale $\propto 2Nf$) will be short compared to those in the other allelic class (on a time-scale $\propto 2Nf(1-f)$). As such, we will ignore coalescent events in the other allelic class. Under these two assumptions as we move away from the selected site, recombinations are events on our genealogy at the selected site that peels lineages out from the selected class allowing them to coalescence on much longer time scales. In Figure \ref{cartoon_fig_1}B we show the coalescent genealogy at the selected site, and show recombination events out of the benefical allele class imposed on the genealogy (these events are colored to correspond to the events on the haplotypes in Figure \ref{cartoon_fig_1}A). In 
Figure \ref{cartoon_fig_1}C we show the genealogy at various points along the sequence shown in Figure \ref{cartoon_fig_1}A, between recombination events. 

Assuming that the frequency of the allele remains fixed at $f$, at a distance $r$ away from the selected site a lineage recombines out of the selected class at rate $r(1-f)$ \gc{Still wonder if we should make this just $r$, as we are assuming $(1-f) \approx 1$ elsewhere}. As coalescent occurs at a rate ${k \choose 2}/(2Nf)$, we can rescale time in units of $2Nf$ so that coalescence happens at a rate ${k \choose 2}$, and recombination events out of the selected class happen at rate $2Nrf(1-f)$. 

If we are interested in the number and size of different recombinant clades at a given distance from the selected site (colored clades in Figure \ref{cartoon_fig_1}B \& C)  this a direct analogy of the infinitely-many allele model \citep{}. In the infinite alleles model, every mutation event creates a new allele, while in our process every recombination event creates a new recombinant lineage (an potentially a distinct haplotype, depending on the configuration of mutations). Under the infinitely many alleles model, sample configurations can be found by simulating the coalescent and assigning current day alleles is found by lineages back through time to the lowest mutation on the genealogy (see Figure \ref{cartoon_fig_1}B). Equivalently we can simulate a sample under the infinitely-many allele model by simulating the mutational process and coalescent process simulatenously, running the process back coalescing lineages and stopping following lineages (`kill' them) when a mutation occurs on them (see Figure \ref{cartoon_fig_1}C). 

Given this nice direct analogy under our set of approximations the number and frequency of the various recombinant lineage classes at a given distance from the selected site can be found using the Ewens' Sampling Formula \citep[ESF][]{}.The population-scaled mutation rate in the infinitely-many alleles model ($\theta/2=4N\mu$), in our model, is replaced by the rate of recombination out of the selected class ($R_{f}/2=4Nrf(1-f)$). If $n$ lineages sampled in the present day fail to recombine off of the selected background during the course of the sweep, then the  probability that these $n$ lineages coalesce into a set of $k$ families and that the number of coalescent families with $1,2,\dots,n$ lineages is given by the partition
 $\{a_1,a_2,\dots,a_n\}$ is 
\begin{equation}
P\left(k,a_1,a_2,\dots,a_n\right) =\frac{R_f^i}{ \prod_{j=1}^{i-1} (R_f +j) } \prod_{j=1}^n\frac{1}{j^{a_j}a_j!} \label{ESF}
\end{equation}
The probability of there being $k$ alleles (recombinant lineages) in our sample of $n$ is 
\begin{equation}
p_{n,k}(R_f)  = S(n,k) \frac{R_f^i}{ \prod_{j=1}^{i-1} (R_f +j) }  \label{ESF2}
\end{equation}
where $S(n,k)$ is a Stirling number of the first kind.  

\subsubsection{Selected phase}
Assume that we are at a neutral site a genetic distance $r$ away from the selected site. 
If we let $X\left(t\right)$ be the frequency of the selected allele at time $t$ in the past, the probability that a single lineage fails to recombine off in generation $t$ is $1-r\left(1-X(t)\right)$. The probability that a single lineage manages to recombine off the selected background at any point during the course of the sweep is given by 
\begin{equation}
P_{NR} \prod_{t=0}^{\tau} 1-r\left(1-X(t)\right)  \approx \exp \left(-r \int_0^{\tau}(1-X\left(t\right))\mathrm{d} t\left(t\right) \right)
\end{equation}
for $r \ll 1$. We set  $\mathcal{T}_{\left(s,f\right)} = \int_0^{\tau}(1-X\left(t\right))\mathrm{d}t$, so that the probability that a lineage manages to recombine off the selected background during the course of the sweep is $e^{-r\mathcal{T}_{\left(s,f\right)}}$. If our selected allele has an additive avantage in terms of relative fitness of $s/2$ in heterozygotes and $s$ in homozygotes, then our allele's deterministic trajectory through the population is logistic and $\mathcal{T}_{\left(s,f\right)} =???$

We assume that the sweep occurs fast enough that the probability of coalescence during the sweep is essentially zero. Therefore, each lineage makes an independent choice of whether it recombines out of the sweep, so that the probability that $i$ out of $k$ lineages fail to escape off the sweep background is
\begin{equation}
P_{NR}(i;k) = {k \choose i} P_{NR}^{i} (1-P_{NR})^{k-i}.
\end{equation}
This binomial approximation has been made by a number of authors in the context of hard sweeps \citep{Barton}, and more accurate approximations have been developed \citep{}. However, as long as the population is large, the sample is not too large,  $\tau$ is not too long, and $f$ not too small (e.g. $\tau {k \choose 2}/(2Nf)<<1$) then this approximation should be adequet. The other more accurate forms could be incorporated into our framework, but we stick with this simple form for the sake of clarity of presentation.

At a given distance away from the selected site every one of the lineages that recombines out of the selected class will be a singleton, and the remaining lineages will be partitioned according to the Ewens' Sampling formula.  

\subsubsection{Patterns of neutral diversity surrounding standing sweeps}
Given this approximate model of the coalescent with a sweep from standing variation we can now calculate basic summaries of variation in the region surrounding the sweep. We will assume that the per base pair mutation rate per generation is $\mu$. We will ignore mutations over the time-scale of our shrunken coalescent tree, and assume that all diversity comes from mutations that occurred prior to the sweep, or equivalently that this part of the genealogy contributes neglibably to the total amount of time in the genealogy. This corresponds to an assumption that $2N\mu \gg 2N \mu f$, in line with our previous set of assumptions that $f \ll 1$. If this is the case we simply consider patterns of diversity in our sample at a site, by considering properties of the recombinant lineages in our sample, which correspond to alleles drawn independently from a neutral population prior to the start of our sweep.

For example, excluding recombination during the sweep for a moment, the expected pairwise coalescent time a distance $r$ away from our sweep is
\begin{equation}
\approx \frac{1}{1 + 4Nrf(1-f)} \times 0 + \frac{4Nrf(1-f)}{1 + 4Nrf(1-f)} \times 2N
\end{equation}
where the two terms correspond to the contribution from failing to recombine during the neutral phase and so coalescing very rapidly, and one or both lineages escaping from the beneficial background and coalesing $2N$ generations ago. In Figure XXX we show this approximation, and coalescent simulations done using $ms$. 

Now incorporating recombination during the sweep our expected pairwise coalescent time a distance $r$ away from our sweep is 
\begin{equation}
\mathbb{E}(T_2) \approx \left(1-\frac{1}{1 + 4Nrf(1-f)} P_{NR}^2  \right) \times 2N
\end{equation}
as to avoid (near) instaneous coalescence our pair of lineages could recombine during either the sweep or neutral phases. The expected level of pairwise diversity as we move away from a sweep is given by $2\mu \mathbb{E}(T_2)$

We can extend this idea of conditioning on how many lineages escape the sweep to calculate the expected total time in the genealogy as we move away from the sweep. Conditional on  $k$ independent lineages escaping the sweep the expected total time in the genealogy is $2N \sum_{j=1}^{k-1} 1/j$, the standard result for a neutral coalescent with $i$ lineages \citep{Watterson}. Ignoring for a moment recombination during the sweep phase the probability that $k$ lineages escape the sweep is the probability of $k$ alleles in a sample of $n$ under the ESF $p_{n,k}(R_f) $ (under our approximation). So the expected time in the genealogy a distance $r$ away from the selected site is 
\begin{equation}
\mathbb{E}(T_{TOT})  \approx 2N \sum_{k=2}^n p_{n,k}(R_f)   \sum_{j=1}^{k-1} 1/j
\end{equation}
In Figure XXX we show this approximation, and coalescent simulations done using $ms$. 

We can now reincorporate recombination during the sweep phase. Under our approximation to recombination during the sweep phase every lineage that recombines out is a singleton. So we can write the probability of having $k$ distinct lineages having recombined out of our sweep as
\begin{equation}
\sum_{i=0}^{k} {n \choose i} P_{NR}^{i} (1-P_{NR})^{n-i} p_{n-i,k-i}(R_f)
\end{equation}
as if we generate $i$ recombinant lineages during our sweep the remaining $k-i$ recombinant lineages have to come from recombination events in the neutral phase. So the expected total time in the genealogy is 
\begin{equation}
\mathbb{E}(T_{TOT})  \approx 2N \sum_{i=0}^{k} {n \choose i} P_{NR}^{i} (1-P_{NR})^{n-i} p_{n-i,k-i}(R_f)   \sum_{j=1}^{k-1} 1/j
\end{equation}
the expected number of segregating sites can be found by taking $\mu$ times this. 




%  In our sample of two, because we assume that the sweep occurs fast enough that the probability of coalescence during the sweep is essentially zero, the two lineages recombine out of the selected phase independently, and thus
% $$P_{norec,s,2} = e^{-2r\mathcal{T}_{\left(s,f\right)}}.$$



% Let's start with the simple case of a sample of two lineages, and work through how to calculate the expected pairwise diversity relative to neutral levels at a given distance form the selected site. We imagine taking a sample of two lineages at the neutral locus the moment the sweep fixes and tracing their genealogy backwards in time. We separate the history of the locus into two different regimes; the first being the time period during which the sweep is occurring, and the second being the period prior to which the allele became beneficial, during which it was either balanced or neutrally drifting at a constant frequency $f$. We make the simplifying assumptions that only recombination (but no coalescence or mutation) can occur during the course of the sweep, while only recombination and coalescence (but no mutation) can occur during the phase when the beneficial allele is polymorphic but neutral in the population. Therefore all diversity arises due to mutations that occurred (forward in time) before the neutral locus came to be associated with the beneficial allele. We will see from simulation results that this is a reasonable assumption. 

% For our purposes it will serve simply to track whether both lineages fail to recombine off of the selected background or not. If either lineage recombines of the selected background, then they coalesce on the standard neutral timescale of $2N$ generations. Otherwise, coalescence during the neutral phase is essentially instantaneous relative to the timescale of mutation, such that we can calculate the expected reduction in pairwise diversity as $1-P_{coal,2}$, where $P_{coal,2}$ is the probability that the two lineages are forced to coalesce before either of them recombines off the background of the beneficial allele. $P_{coal,2}$ can be obtained as $P_{norec,s,2}P_{coalfirst,n,2}$ \jb{I don't know what to call these things in any compact yet readable notation}, where $P_{norec,s,2}$ is the probability that neither lineage recombines off the selected background during the course of the sweep, and $P_{coalfirst,n,2}$ is the probability that the two lineages coalesce during the neutral phase before either of them recombines off the background of the beneficial allele, assuming neither recombined off during the sweep. We'll treat $P_{norec,s,2}$ first and $P_{coalfirst,n,2}$ after that.

% If we let $X\left(t\right)$ be the frequency of the selected allele at time $t$ in the past, the probability that a single lineage fails to recombine off in generation $t$ is $e^{-r\left(1-X(t)\right)}$. The probability that a single lineage manages to recombine off the selected background at any point during the course of the sweep is given by $e^{-r \int_f^{1}(1-X\left(t\right))\mathrm{d}X\left(t\right)}$. For now I will simply set $\mathcal{T}_{\left(s,f\right)} = \int_f^{1}(1-X\left(t\right))\mathrm{d}X\left(t\right)$, so that the probability that a lineage manages to recombine off the selected background during the course of the sweep is $e^{-r\mathcal{T}_{\left(s,f\right)}}$. In our sample of two, because we assume that the sweep occurs fast enough that the probability of coalescence during the sweep is essentially zero, the two lineages recombine out of the selected phase independently, and thus
% $$P_{norec,s,2} = e^{-2r\mathcal{T}_{\left(s,f\right)}}.$$

% Now let's consider the second phase, in which the (soon to be) beneficial allele is either balanced or neutral. We make the simplifying assumption that if the frequency of the allele was $f$ when the sweep began, then it is held constant at $f$ for the time period that is relevant to the processes we are interested in. Because the two alleles exist within a small subpopulation with size $2Nf$, they coalesce at rate $\frac{1}{2Nf}$. Each lineage at the neutral locus recombines off the background of the (future) selected allele at a rate $r(1-f)$. Therefore, the probability that the two lineages coalesce before either of them manages to recombine off the background is given by
% $$P_{coalfirst,n,2} = \frac{\frac{1}{2Nf}}{\frac{1}{2Nf} + r(1-f)} = \frac{1}{1 + 4Nrf(1-f)}.$$

% Taking these together we find that the expected pairwise diversity at distance $r$ away from a sweep from standing variation that begins from frequency $f$ is approximately equal to
% $$\pi_{r,s,f} = \pi_0\left(1 - \frac{e^{-2r\mathcal{T}_{\left(s,f\right)}}}{1 + 4Nrf\left(1-f\right)}\right).$$

% Examining the formula for $P_{coalfirst,n,2}$, we notice that it is very similar to the formula for the probability that two lineages chosen randomly from a population are of the same allelic type under the Ewens Sampling Formula, which is given by $\frac{1}{1+4N\mu}$. Specifically, we've just replaced the mutation parameter $\mu$ in the standard Ewens Sample Formula with the compound parameter $rf\left(1-f\right)$, which can be though of as the per generation probability that a randomly chosen lineage recombines off of the background of the beneficial allele onto the other background (there is a probability $f$ that the randomly chosen lineage is currently on the beneficial background, and given that it is, a probability $r\left(1-f\right)$ that it recombines onto the other background). For ease of bookkeeping, we will let $R_f = 4Nrf\left(1-f\right)$.

% The observation regarding the the Ewens Sampling Formula above is in fact general, in the sense that if $n$ lineages sampled in the present day fail to recombine off of the selected background during the course of the sweep, then the  probability that these $n$ lineages coalesce into a set of $k$ families and that the number of coalescent families with $1,2,\dots,n$ lineages is given by the partition $\{a_1,a_2,\dots,a_n\}$ is 
% $$P\left(k,a_1,a_2,\dots,a_n\right) = \frac{n!R_f^k}{{R_f}_{\left(n\right)}}\prod_{j=1}^n\frac{1}{j^{a_j}a_j!}$$
% where ${R_f}_{\left(n\right)} = \prod_{i=0}^{n-1}\left(R_f+i\right)$.

\section{Discussion}


%%%%%%%%%%%%%%%%%%%%%%%%%%%%
\section{Acknowledgements}

\section{Methods}


\bibliographystyle{genetics}
\bibliography{library,morelibrary}

\section{Supplementary materials}

\setcounter{table}{0}
\renewcommand{\thetable}{S\arabic{table}}
\setcounter{figure}{0}
\renewcommand{\thefigure}{S\arabic{figure}}

\end{document}
